\section{Introducción}

En este trabajo se propone una solución para resolver el problema de encontrar árboles generadores mínimos de grafos (AGM) con múltiples threads, evitando deadlocks y otros problemas asociados. El algoritmo tradicional es Prim ejecutado como un único thread en un proceso, empezando en un nodo y avanzando siempre por el eje de menor peso hacia el próximo vecino no visitado.

Nuestra propuesta permite elegir la cantidad de threads que van a encontrar un AGM para un grafo determinado, al momento de ejecución se utiliza una versión paralela del antes mencionado algoritmo de Prim, replicado en tantos hilos como hayan sido elegidos el mismo algoritmo pero iniciando cada uno desde un nodo diferente del mismo grafo.

Los problemas potenciales que pueden existir en esta estrategia surgen cuando más de un thread quiere capturar el mismo nodo, o cuando uno o más threads quiere capturar un nodo perteneciente a otro thread. A estos escenarios se suma la complejidad de resolver ciclos cuando dos threads se quieren fusionar mutuamente, y aquellos relacionados con la naturaleza de la programación asincrónica y distribuida: evitar deadlocks, starvations, race conditions y asegurar la contención. Un esquema simple para resolver estas dificultades fue propuesto utilizando locks y trylocks de la libreria Pthreads.

Se presentan resultados de experimentaciones para analiza el método propuesto.

\subsection{AGM}

El problema del Árbol Generador Mínimo fue resuelto por primera vez en 1926 día de hoy tiene numerosas aplicaciones en organización, ruteo y partición de redes. La definición del problema es la siguiente: dado un grafo conexo y no dirigido G con pesos en sus ejes, encontrar el árbol generador de peso total mínimo. Es decir, encontrar un subgrafo que sea un árbol, que contenga a todos los vértices de G y tal que la suma de los pesos de los ejes del subgrafo sea igual o menor a la de cualquier otro árbol generador de G.

Para resolver este problema utilizamos una variante del algoritmo de Prim. La misma es parametrizable en la cantidad de threads, permitiendo la búsqueda del AGM desde distintos nodos del grafo en diferentes hilos de ejecución.

\subsection{Programación paralela y Pthreads}

Un proceso es un programa en ejecución. El objetivo de utilizar paralelismo es optimizar la resolución de un problema, partiéndolo en unidades más pequeñas que puedan ejecutarse en paralelo. Para esto utilizamos threads (con la librería pthreads), una alternativa más eficiente a múltiples procesos (por medio de forks).

Al utilizar threads se comparte el mismo espacio de direcciones, lo cual facilita compartir datos entre diferentes hilos. De esta manera, la estructura del grafo y sus 'colores' son compartidos, cada thread crece su propio agm, pintando cada nodo del mismo con su ID. Es necesario garantizar la contención en los casos que distintos hilos quieran acceder a las mismas estructuras en el mismo momento. Para esto hacen falta ciertas estructuras para la sincronización que serán explicadas en las siguientes secciones del informe.
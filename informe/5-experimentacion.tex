\section{Experimentación}
\subsection{Mediciones y metodología de experimentación}

Para poder realizar las mediciones correspondientes a los experimentos realizados utilizamos la instrucción de assembler \textit{rdtsc}. Con ella podemos obtener el valor de Time Stamp Counter (TSC) del procesador -un registro de 64 bits presente en todos los procesadores x86 desde los Pentium-, como este se incrementa en uno con cada ciclo del procesador podemos obtener la duración en cantidad de ticks de una llamada a una función calculando la diferencia de este registro al principio y al final de su ejecución. Ya que esta medida no es constante entre llamadas, por cada caso de test (ejecución de un filtro, con una implementación y para una imagen con un tamaño específico) se realizaron 2500 llamadas. Al mismo tiempo, para poder tener las mediciones individuales de cada iteración y poder calcular la varianza y filtrar outliers se modificó el framework de la cátedra para tener disponible esta información.
La modificación se realizó en el ciclo principal de la función \textit{correr_filtro_imagen} del archivo \textit{tp2.c}:

\begin{codesnippet}
  \begin{verbatim}
    MEDIR_TIEMPO_START(start)
    for (int i = 0; i < config->cant_iteraciones; i++) {
        unsigned long long start_iter, end_iter, iteraciones;
        MEDIR_TIEMPO_START(start_iter)
        aplicador(config);
        MEDIR_TIEMPO_STOP(end_iter)
        iteraciones = end_iter - start_iter;
        fprintf(fp, ";%llu", iteraciones);
    }
    MEDIR_TIEMPO_STOP(end)
  \end{verbatim}
\end{codesnippet}

Luego, una vez que se contó con los tiempos de cada iteración se lidió con los outliers calculando una media 25\% podada. La computadora en la cual se corrieron los experimentos tiene las siguientes especificaciones:

\begin{table}[htb]
  \centering
  \caption{Especificaciones cpu}
  \begin{tabular}{|c|c|}
  \hline
    model name & Intel(R) Core(TM) 14-nm "Broadwell ULT" i5-5350U CPU @ 1.80GHz \\ \hline
    cpu MHz & 1799.948 \\ \hline
    l2 cache size & 256 KB \\ \hline
    l3 cache size & 3072 KB \\ \hline
    mem total & 8 GB 1600 MHz DDR3 \\ \hline
  \hline
  \end{tabular}
  \end{table}

A su vez, cuenta con los flags detallados a continuación: 
fpu vme de pse tsc msr pae mce cx8 apic sep mtrr pge mca cmov pat pse36 clflush mmx fxsr sse sse2 ht syscall nx rdtscp lm constant_tsc rep_good nopl xtopology nonstop_tsc eagerfpu pni pclmulqdq monitor ssse3 cx16 pcid sse4_1 sse4_2 movbe popcnt aes xsave avx rdrand lahf_lm abm 3dnowprefetch invpcid_single fsgsbase avx2 invpcid rdseed flush_l1d \\

Se utilizó la siguiente combinación de sistema operativo y compiladores: Ubuntu/Xenial64 16.04.4 LTS, GCC versión 5.4.0 y NASM versión 2.11.08. Para compilar el código C se usaron los flags \textit{-ggdb -Wall -Wno-unused-parameter -Wextra -std=c99 -pedantic -m64 -O0} y para ASM \textit{-felf64 -g -F dwarf} conforme a los archivos proporcionados por la cátedra de la materia.

La imagen utilizada para todos los experimentos fue la correspondiente a \textit{No Country For Old Men}, proporcionada por la cátedra. Otros detalles de las mediciones se dan específicamente por cada experimento producido.
% ******************************************************** %
%              TEMPLATE DE INFORME ORGA2 v0.1              %
% ******************************************************** %
% ******************************************************** %
%                                                          %
% ALGUNOS PAQUETES REQUERIDOS (EN UBUNTU):                 %
% ========================================
%                                                          %
% texlive-latex-base                                       %
% texlive-latex-recommended                                %
% texlive-fonts-recommended                                %
% texlive-latex-extra?                                     %
% texlive-lang-spanish (en ubuntu 13.10)                   %
% ******************************************************** %
\documentclass[a4paper]{article}
\usepackage[spanish]{babel}
\usepackage[utf8]{inputenc}
\usepackage{charter}   % tipografia
\usepackage{graphicx}
%\usepackage{makeidx}
\usepackage{paralist} %itemize inline
\usepackage{algorithm}  % implementacion ondas en C
% sudo apt-get install texlive-science
\usepackage{algorithmic} % implementacion ondas en C
%\usepackage{float}
\usepackage{amsmath, amsthm, amssymb}
%\usepackage{amsfonts}
%\usepackage{sectsty}
%\usepackage{charter}
%\usepackage{wrapfig}
%\usepackage{listings}
%\lstset{language=C}
% \setcounter{secnumdepth}{2}
\usepackage{wrapfig}
\usepackage{underscore}
\usepackage{caratula}
\usepackage{url}
\usepackage{multicol}
\usepackage{algpseudocode}

% ********************************************************* %
% ~~~~~~~~              Code snippets             ~~~~~~~~~ %
% ********************************************************* %
\usepackage{color} % para snipets de codigo coloreados
\usepackage{fancybox}  % para el sbox de los snipets de codigo
\definecolor{litegrey}{gray}{0.94}
\newenvironment{codesnippet}{%
	\begin{Sbox}\begin{minipage}{\textwidth}\sffamily\small}%
	{\end{minipage}\end{Sbox}%
		\begin{center}%
		\vspace{0cm}\colorbox{litegrey}{\TheSbox}\end{center}\vspace{0cm}}
% ********************************************************* %
% ~~~~~~~~         Formato de las páginas         ~~~~~~~~~ %
% ********************************************************* %
\usepackage{fancyhdr}
\pagestyle{fancy}
%\renewcommand{\chaptermark}[1]{\markboth{#1}{}}
\renewcommand{\sectionmark}[1]{\markright{\thesection\ - #1}}
\fancyhf{}
\fancyhead[LO]{Sección \rightmark} % \thesection\ 
\fancyfoot[LO]{\small{Centeno, Romera, Ghianni}}
\fancyfoot[RO]{\thepage}
\renewcommand{\headrulewidth}{0.5pt}
\renewcommand{\footrulewidth}{0.5pt}
\setlength{\hoffset}{-0.8in}
\setlength{\textwidth}{16cm}
%\setlength{\hoffset}{-1.1cm}
%\setlength{\textwidth}{16cm}
\setlength{\headsep}{0.5cm}
\setlength{\textheight}{25cm}
\setlength{\voffset}{-0.7in}
\setlength{\headwidth}{\textwidth}
\setlength{\headheight}{13.1pt}
\renewcommand{\baselinestretch}{1.1}  % line spacing
% ******************************************************** %

\begin{document}

\thispagestyle{empty}
\materia{Sistemas Operativos}
\submateria{Segundo Cuatrimestre de 2019}
\titulo{Trabajo Práctico I}
\integrante{Joaquin P. Centeno}{699/16}{joaquinpcenteno@gmail.com}
\integrante{Joaquin Romera }{183/16}{joakromera@gmail.com}
\integrante{Hernán Ghianni}{538/16}{herghia@gmail.com}

\maketitle

\newpage

\thispagestyle{empty}
\vfill
\thispagestyle{empty}
\vspace{3cm}

\tableofcontents
\newpage

\section{Introducción}
En este trabajo se propone una solución para resolver el problema de encontrar árboles generadores mínimos de grafos (AGM) con múltiples threads, evitando deadlocks y otros problemas asociados a los algoritmos sobre sistemas ditribuídos. El algoritmo tradicional es Prim ejecutado como un único proceso, empezando en un nodo y avanzando siempre por el eje de menor peso hacia el próximo vecino no visitado. Nuestra propuesta permite elegir la cantidad de threads que van a encontrar un AGM para un grafo determinado, al momento de ejecución se utiliza una versión paralela del antes mencionado algoritmo de Prim, replicado en tantos hilos como hayan sido elegidos el mismo proceso pero iniciando cada uno desde un nodo diferente del mismo grafo. Los problemas potenciales que pueden existir en esta estrategia surgen cuando más de un thread quiere capturar el mismo nodo, o cuando uno o más threads quiere capturar un nodo perteneciente ya a otro thread. A estos escenarios se suma la complejidad de resolver ciclos cuando dos threads se quieren fusionar mutuamente, y aquellso relacionados con la naturaleza de la programación asincrónica y distribuida: evitar deadlocks, starvations, y race conditions. Un esquema simple para resolver estas dificultades fue propuesta utilizando locks y trylocks. Se presentan resultados de experimentaciones para mostrar la efectividad del método propuesto.

% \begin{center}
%   \begin{tabular}{cccc}
%     \includegraphics[width=0.30\textwidth]{imagenes/sisdsimd.png} &
%     \includegraphics[width=0.30\textwidth]{imagenes/registrossimd.png}\\
%   \end{tabular}
%  \end{center}

\subsection{AGM}

% \begin{center}
%  \begin{tabular}{cccc}
%    \includegraphics[width=0.2\textwidth]{imagenes/ncfom.jpg} &
%    \includegraphics[width=0.2\textwidth]{imagenes/ncfom-cuadrados.jpg} &
%    \includegraphics[width=0.2\textwidth]{imagenes/ncfom-manchas.jpg} \\
%    Imagen original & Cuadrados & Manchas \\
%    \\
%    \includegraphics[width=0.2\textwidth]{imagenes/ncfom-offset.jpg} &
%    \includegraphics[width=0.2\textwidth]{imagenes/ncfom-sharpen.jpg} \\
%    Offset & Sharpen \\
%  \end{tabular}
% \end{center}

\subsection{Programación paralela y Pthreads}

\newpage
\section{Desarrollo}

\subsection{Estructuras}

\subsubsection{Thread}
\begin{verbatim}
struct Thread{
    Grafo agm;
    ColaDePrioridad ejesVecinos;
    pthread_mutex_t fusion_req;   
    pthread_mutex_t fusion_ack;   
    pthread_mutex_t fusion_ready; 
};
\end{verbatim}

 Trabajamos con esta estructura para resguardar la información de cada thread. La misma consiste de un grafo que será el agm parcial del thread, una cola de prioridad que contiene a los ejes cuyo nodo destino es alcanzable desde el agm y tres mutex para administrar las fusiones. Tendremos un vector global que contendrá uno de estos structs por cada thread. El nombre del mismo es threadData.



\subsubsection{Colores}
\begin{verbatim}
class Colores {

    vector<int> _colores;
    vector<pthread_mutex_t> _locks;
};
\end{verbatim}
Para controlar que thread es dueño de cada nodo utilizamos esta estructura que contiene un vector con el color de cada uno, los cuales no son otra cosa que los thread id. Para evitar race condition e incongruencias cada nodo tiene además un mutex asociado que utilizaremos tanto para conocer a su dueño como para cambiarlo. 

\subsubsection{Fusión}

Como dijimos cada thread cuenta con tres mutex para administrar tanto la solicitud como la entrada de fusiones. En cada ciclo los threads van a chequear si tienen funciones entrantes haciendo trylock de su mutex fusion$\textunderscore$req. Mientras no puedan tomarlo implica que otro thread esta esperando para fusionarse. De ser así hace un unlock de su fusion$\textunderscore$ack despertando al thread solicitante y un lock de fusion$\textunderscore$ready. Esto ultimo es para esperar a que el otro thread termine de fusionarse. Una vez que ya no tiene pendientes procede a buscar al próximo nodo. En caso de que pertenezca a otro thread solicita una fusión de la siguiente forma: Primero chequea que no tenga funciones entrantes haciendo un trylock de su fusion$\textunderscore$req. Esto lo hacemos para evitar deadlock. Luego lo mismo pero con el fusion$\textunderscore$req del thread dueño del nodo que quiere conseguir. Si puede conseguirlo hace lock del fusion$\textunderscore$ack del otro thread quedando así en espera hasta que el mismo este preparado para realizar la fusión. Una vez que la misma se produce se vuelve a chequear al dueño del nodo ya que podría haber ocurrido una fusión y por lo tanto no ser el mismo. Si se mantuvo se debe chequear cual es el de menor id que será el que "sobreviva". El otro sera reiniciado y comenzará de cero. El sobreviviente tendrá ahora la unión de los agm y las colas de prioridad. Por ultimo ambos fusion$\textunderscore$req serán desbloqueados y así como el fusion$\textunderscore$ready del perdedor. Es importante remarcar que el thread reiniciado mantiene el estado de sus mutex ya que sino podría ocurrir que algún thread quede colgado esperando fusionarse. Si no hubiera sido posible conseguir alguno de los fusion$\textunderscore$req o el dueño del nodo hubiera cambiado el thread habría vuelto al principio del ciclo.

\subsection{Algoritmo}
\begin{algorithm}
\begin{algorithmic}
\caption{mstParaleloThread}
\Function{mstParaleloThread}{grafo}

\STATE $my$\textunderscore$id \gets thread$\textunderscore$counter++ $
\STATE $eje$\textunderscore$actual$
\STATE $my$\textunderscore$data \gets threadData[my$\textunderscore$id]$
\WHILE{$true$}
\STATE $thread$\textunderscore$attend$\textunderscore$fusion$\textunderscore$requests(my$\textunderscore$id)$
\STATE $my$\textunderscore$data.fusion$\textunderscore$req.unlock()$
\STATE $status \gets buscarNodo(my$\textunderscore$id,ejeActual)$
\IF{($status = noHayNodosDisponibles$)}
\STATE $thread$\textunderscore$attend$\textunderscore$fusion$\textunderscore$requests(my$\textunderscore$id)$
\ENDIF
\IF{($status = agmCompleto$)}
\STATE $break$
\ENDIF
\STATE $owner$\textunderscore$id \gets colores.capturarNodo(ejeActual.nodoDestino,my$\textunderscore$id)$
\IF{($my$\textunderscore$id = owner$\textunderscore$id$)}
\STATE $my$\textunderscore$data.insertarEje(ejeActual)$
\STATE $my$\textunderscore$data.agregarEjesVecinos(g,ejeActual.nodoDestino)$
\ELSE 
\STATE $owner$\textunderscore$data \gets threadData[owner$\textunderscore$id]$
\IF{($my$\textunderscore$data.fusion$\textunderscore$req.trylock() \neq 0 )$}
\STATE $continue$
\ENDIF
\IF{($owner$\textunderscore$data.fusion$\textunderscore$req.trylock() \neq 0$)}
\STATE $my$\textunderscore$data.fusion$\textunderscore$req.unlock()$
\STATE $continue$
\ENDIF
\STATE $owner$\textunderscore$data.fusion$\textunderscore$ack.lock()$
\IF{($colores.esDueño(ejeActual,owner$\textunderscore$id)$)}
\STATE $my$\textunderscore$data.agm.insertarEje(ejeActual)$
\IF{($my$\textunderscore$id < owner$\textunderscore$id $)}
\STATE $fuse(my$\textunderscore$id,owner$\textunderscore$id)$
\ELSE
\STATE $fuse(owner$\textunderscore$id,my$\textunderscore$id)$
\ENDIF
\ENDIF
\ENDIF
\STATE $owner$\textunderscore$id.fusion$\textunderscore$req.unlock()$
\STATE $my$\textunderscore$id.fusion$\textunderscore$req.unlock()$
\STATE $owner$\textunderscore$id.fusion$\textunderscore$ready.unlock()$
\ENDWHILE
\end{algorithmic}
\end{algorithm}


\newpage
Aclaraciones:

\begin{itemize}
    \item No realizamos copias del threadData en la implementación. Esto es solamente para clarificar el pseudocódigo
\end{itemize}

\newpage
\input{2-impl}
\newpage
\section{Experimentación}
\subsection{Mediciones y metodología de experimentación}

\subsection{Performance: diferentes tamaños de grafos}

\subsection{Performance: diferentes densidad de aristas}

\subsection{Performance: diferentes cantidad de threads}

% \begin{codesnippet}
%   \begin{verbatim}
%     MEDIR_TIEMPO_START(start)
%     for (int i = 0; i < config->cant_iteraciones; i++) {
%         unsigned long long start_iter, end_iter, iteraciones;
%         MEDIR_TIEMPO_START(start_iter)
%         aplicador(config);
%         MEDIR_TIEMPO_STOP(end_iter)
%         iteraciones = end_iter - start_iter;
%         fprintf(fp, ";%llu", iteraciones);
%     }
%     MEDIR_TIEMPO_STOP(end)
%   \end{verbatim}
% \end{codesnippet}


% \begin{table}[htb]
%   \centering
%   \caption{Especificaciones cpu}
%   \begin{tabular}{|c|c|}
%   \hline
%     model name & Intel(R) Core(TM) 14-nm "Broadwell ULT" i5-5350U CPU @ 1.80GHz \\ \hline
%     cpu MHz & 1799.948 \\ \hline
%     l2 cache size & 256 KB \\ \hline
%     l3 cache size & 3072 KB \\ \hline
%     mem total & 8 GB 1600 MHz DDR3 \\ \hline
%   \hline
%   \end{tabular}
% \end{table}

\newpage
\input{4-analisis}
\newpage
\section{Conclusiones y trabajo futuro}

Gracias a los experimentos realizados en el presente trabajo, se pudo llegar a la conclusión de que las ventajas que puede brindar el paradigma \textbf{SIMD} a la hora de implementar programas que realicen operaciones altamente paralelizables, como el procesamiento de imágenes, son verdaderamente significativas. Esto queda reflejado en la gran brecha de rendimiento que se observa entre algunas implementaciones realizadas con dicho paradigma y las que utilizan el lenguaje de programación C.

Esto siempre debe contraponerse a otro hecho que se hizo presente durante el proceso de implementación: realizar un programa en lenguaje ensamblador resulta, por lo general, considerablemente más difícil que hacerlo en un lenguaje de más alto nivel. El código resultante es menos legible, es más sencillo cometer errores y el proceso de \emph{debugging} se vuelve considerablemente más arduo. Por eso es importante analizar de antemano las características del contexto particular de aplicación, para poder decidir si este esfuerzo adicional realmente vale la pena.

Incluso una vez realizada una implementación en ASM, es necesario comparar el rendimiento de ambas versiones con distintos tamaños de imágenes para ver que efectivamente haya una mejora y el grado de la misma -por ejemplo, en nuestras implementaciones Offset para tamaños grandes obtuvo peores resultados que la versión en C con O3-. Otros aspectos a tener en consideración es el tamaño de los binarios involucrados, que puede ser decisivo en determinados escenarios.

También se intentó la realización de una optimización manual dentro del código ensamblador, con distintos resultados. Por ejemplo: expander los ciclos, inlinear llamadas a funciones, analizar instrucciones individuales para encontrar alternativas, tuvieron distintos beneficios en algunos casos y en otros no. Estas optimizaciones son realizadas por el compilador de C al setear alguna opción del flag O, las distintas técnicas de optimización que realizan los mismos pueden servir de inspiración a la hora de intentar mejorar el rendimiento de un programa, pero su justificación siempre tendrá que ser acompañada de métricas que las respalden.

Por último, pudimos hacer una pequeña prueba con la extensión del modelo de SIMD propuesto por la tecnología AVX2 cuyos resultados confirmaron las ventajas de su utilización en la programación vectorial a bajo nivel.

Esperamos que la experiencia nos provea de nuevos recursos a la hora de afrontar problemáticas similares en el futuro.
\end{document}

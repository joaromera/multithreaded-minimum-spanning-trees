% ******************************************************** %
%              TEMPLATE DE INFORME ORGA2 v0.1              %
% ******************************************************** %
% ******************************************************** %
%                                                          %
% ALGUNOS PAQUETES REQUERIDOS (EN UBUNTU):                 %
% ========================================
%                                                          %
% texlive-latex-base                                       %
% texlive-latex-recommended                                %
% texlive-fonts-recommended                                %
% texlive-latex-extra?                                     %
% texlive-lang-spanish (en ubuntu 13.10)                   %
% ******************************************************** %
\documentclass[a4paper]{article}
\usepackage[spanish]{babel}
\usepackage[utf8]{inputenc}
\usepackage{charter}   % tipografia
\usepackage{graphicx}
%\usepackage{makeidx}
\usepackage{paralist} %itemize inline
\usepackage{algorithm}  % implementacion ondas en C
% sudo apt-get install texlive-science
\usepackage{algorithmic} % implementacion ondas en C
%\usepackage{float}
\usepackage{amsmath, amsthm, amssymb}
%\usepackage{amsfonts}
%\usepackage{sectsty}
%\usepackage{charter}
%\usepackage{wrapfig}
%\usepackage{listings}
%\lstset{language=C}
% \setcounter{secnumdepth}{2}
\usepackage{wrapfig}
\usepackage{underscore}
\usepackage{caratula}
\usepackage{url}
\usepackage{multicol}

% ********************************************************* %
% ~~~~~~~~              Code snippets             ~~~~~~~~~ %
% ********************************************************* %
\usepackage{color} % para snipets de codigo coloreados
\usepackage{fancybox}  % para el sbox de los snipets de codigo
\definecolor{litegrey}{gray}{0.94}
\newenvironment{codesnippet}{%
	\begin{Sbox}\begin{minipage}{\textwidth}\sffamily\small}%
	{\end{minipage}\end{Sbox}%
		\begin{center}%
		\vspace{0cm}\colorbox{litegrey}{\TheSbox}\end{center}\vspace{0cm}}
% ********************************************************* %
% ~~~~~~~~         Formato de las páginas         ~~~~~~~~~ %
% ********************************************************* %
\usepackage{fancyhdr}
\pagestyle{fancy}
%\renewcommand{\chaptermark}[1]{\markboth{#1}{}}
\renewcommand{\sectionmark}[1]{\markright{\thesection\ - #1}}
\fancyhf{}
\fancyhead[LO]{Sección \rightmark} % \thesection\ 
\fancyfoot[LO]{\small{Centeno, Romera, Ghianni}}
\fancyfoot[RO]{\thepage}
\renewcommand{\headrulewidth}{0.5pt}
\renewcommand{\footrulewidth}{0.5pt}
\setlength{\hoffset}{-0.8in}
\setlength{\textwidth}{16cm}
%\setlength{\hoffset}{-1.1cm}
%\setlength{\textwidth}{16cm}
\setlength{\headsep}{0.5cm}
\setlength{\textheight}{25cm}
\setlength{\voffset}{-0.7in}
\setlength{\headwidth}{\textwidth}
\setlength{\headheight}{13.1pt}
\renewcommand{\baselinestretch}{1.1}  % line spacing
% ******************************************************** %

\begin{document}

\thispagestyle{empty}
\materia{Sistemas Operativos}
\submateria{Segundo Cuatrimestre de 2019}
\titulo{Trabajo Práctico I}
\integrante{Joaquin P. Centeno}{699/16}{joaquinpcenteno@gmail.com}
\integrante{Joaquin Romera }{183/16}{joakromera@gmail.com}
\integrante{Hernán Ghianni}{538/16}{herghia@gmail.com}

\maketitle

\newpage

\thispagestyle{empty}
\vfill
\thispagestyle{empty}
\vspace{3cm}

\tableofcontents
\newpage

\section{Introducción}
En el presente trabajo hicimos un primer acercamiento al modelo de ejecución SIMD (Single instruction Multiple Data). Diseñado como una extensión al set de instrucciones de la arquitectura x86 e introducido por Intel en 1999, esta técnica de paralelización puede mejorar notablemente la performance de un sistema en contextos de programación donde se deben aplicar algoritmos repetitivos sobre un mismo conjunto de datos. Esto lo hace particularmente útil para procesamientos multimedia -audio, video e imágenes- donde la reproducción en tiempo real es crítica. Gracias a las instrucciones SIMD podemos ejecutar una misma operación sobre muchos datos al mismo tiempo en una sola instrucción, contrario a realizarlo en un modelo exclusivamente SISD (single instruction single data). Como veremos a lo largo del trabajo su utilización introducirá notables mejoras en performance y eficiencia de nuestros algoritmos en una gran cantidad de casos siempre que sea posible su aplicación.

\begin{center}
  \begin{tabular}{cccc}
    \includegraphics[width=0.30\textwidth]{imagenes/sisdsimd.png} &
    \includegraphics[width=0.30\textwidth]{imagenes/registrossimd.png}\\
  \end{tabular}
 \end{center}

\subsection{Objetivos Generales}
Además de la exploración del modelo SIMD se repasaron conceptos y técnicas de programación vectorizada en C y ASM dentro del campo de aplicación del procesamiento de imágenes. Para esto se llevó a cabo la implementación de los siguientes filtros:

\begin{center}
 \begin{tabular}{cccc}
   \includegraphics[width=0.2\textwidth]{imagenes/ncfom.jpg} &
   \includegraphics[width=0.2\textwidth]{imagenes/ncfom-cuadrados.jpg} &
   \includegraphics[width=0.2\textwidth]{imagenes/ncfom-manchas.jpg} \\
   Imagen original & Cuadrados & Manchas \\
   \\
   \includegraphics[width=0.2\textwidth]{imagenes/ncfom-offset.jpg} &
   \includegraphics[width=0.2\textwidth]{imagenes/ncfom-sharpen.jpg} \\
   Offset & Sharpen \\
 \end{tabular}
\end{center}

\subsection{Metodología de trabajo}
Al disponer de las implementaciones en C de todos los filtros, se procedió a realizar su implementación en lenguaje ensamblador para la arquitectura x86-64 de Intel. Para esto, se utilizaron las instrucciones SSE de dicha arquitectura, que aprovechan el ya mencionado modelo SIMD para procesar datos en forma paralela.

Una vez realizadas estas implementaciones, fueron sometidas a un proceso de comparación para extraer conclusiones acerca de su rendimiento. Con este fin, se experimentó con variaciones tanto en los datos de entrada como en detalles implementativos de los mismos algoritmos. De esta manera, se pudo recopilar datos sobre el comportamiento de cada implementación, y contrastar estos resultados con diversas hipótesis previamente elaboradas.

A continuación introducimos los filtros y sus respectivas implementaciones, luego describimos los tests realizados y los resultados obtenidos, y finalmente a partir de estos datos otorgamos algunas conclusiones sobre la experiencia realizada.
\newpage

\section{Filtros}
\subsection{Cuadrados}

\subsubsection{Descripción}

\begin{center}
	\includegraphics[width=0.5\textwidth]{imagenes/ncfom-cuadrados.jpg}
\end{center}

Esta operación genera un efecto \textit{geométrico}, borrando curvas y generando un efecto de \textit{cuadrados}. Esto se logra reemplazando cada píxel por el máximo en un cuadrado con vértice en dicho píxel.
\[cuadrado(p) = \underset{0\leq i \leq 3, 0 \leq j \leq 3}{\max} A[p_0+i][p_1+j] \]

\subsubsection{Implementación ASM}

En este filtro procesamos de a 4 píxeles: tenemos dos ciclos, uno que recorre las columnas de la imagen y otro que recorre las filas. A continuación el pseudocódigo de la implementación \textit{SIMD}:

\begin{itemize}

\item Primero se arma el stack frame y se guardan en los registros de propósito general las entradas. (Líneas 30 a 45)
\item Luego se dibuja el borde negro de 4 píxeles con el llamado a la función auxiliar \textit{CompletarConCeros} en la línea 52.
\item Antes de comenzar el ciclo se limpian los registros usados para evitar errores. (Líneas 62 a 70)
\item Se setea el comienzo de las imágenes fuente y de destino luego del borde negro, y se trae de memoria la máscara a usar. (Líneas 72 a 77)
\item Iniciamos los índices de columnas, filas y offset a usar para los ciclos del algoritmo. (Líneas 80 a 91)
\item Condiciones de los ciclos de filas y columnas: si la fila llegó al final, saltar al final del ciclo de filas, y análogo con las columnas.
\item El cuerpo del ciclo principal del filtro es donde se realiza el proceso del filtro. En los registros XMM se carga la matriz desde la memoria.
\end{itemize}

\begin{codesnippet}
  \begin{verbatim}
	103		;Loading 4x4 matrix on XMM registers.
	104		movdqu first_row, [src]
	105		movdqu second_row, [src+src_row_size]
	106		movdqu third_row, [src+src_row_size*2]
	107		movdqu fourth_row, [src+fourth_row_offset]
  \end{verbatim}
\end{codesnippet}

\begin{center}
  \begin{tabular}{|c || c || c || c |}
  \hline
  $row1[0]$ & $row1[1]$ & $row1[2]$ & $row1[3]$ \\ \hline
  \end{tabular}
  \\ \textbf{XMM0 $\gets$ row1}
\end{center}
\begin{center}
  \begin{tabular}{|c || c || c || c |}
  \hline
  $row2[0]$ & $row2[1]$ & $row2[2]$ & $row2[3]$ \\ \hline
  \end{tabular}
  \\ \textbf{XMM1 $\gets$ row2}
\end{center}
\begin{center}
  \begin{tabular}{|c || c || c || c |}
  \hline
  $row3[0]$ & $row3[1]$ & $row3[2]$ & $row3[3]$ \\ \hline
  \end{tabular}
  \\ \textbf{XMM2 $\gets$ row3}
\end{center}
\begin{center}
  \begin{tabular}{|c || c || c || c |}
  \hline
  $row4[0]$ & $row4[1]$ & $row4[2]$ & $row4[3]$ \\ \hline
  \end{tabular}
  \\ \textbf{XMM3 $\gets$ row4}
\end{center}

\begin{itemize}
\item Una vez que el algoritmo carga la matriz en los registros XMM0-XMM3, aplica el algoritmo para hallar el máximo. Este se encontrara en la parte baja del registro XMM4, y se copia a memoria en la imagen de destino.
\end{itemize}

\begin{codesnippet}
  \begin{verbatim}
	;Find maximums and store in the vector_maximum register.
	110		jmp .hallarMaximos

	112		.retornarDeMaximos:
			;Save maximums on the destination image.
	114		movss [dst], vector_maximum
  \end{verbatim}
\end{codesnippet}

\begin{itemize}
\item Al final de estas cuatro comparaciones, el vector maximum tendrá en cada píxel, el máximo de su correspondiente columna.
\end{itemize}

\begin{codesnippet}
  \begin{verbatim}
			;Algorithm for finding maximums:
	137		.hallarMaximos:
	138		pxor vector_maximum, vector_maximum

			;Find maximum comparing row by row on each column.
	141		pmaxub vector_maximum, first_row
	142		pmaxub vector_maximum, second_row
	143		pmaxub vector_maximum, third_row
	144		pmaxub vector_maximum, fourth_row
  \end{verbatim}
\end{codesnippet}

\begin{center}
  \begin{tabular}{|c || c || c || c |}
  \hline
  $max(Column1)$ & $max(Column2)$ & $max(Column3)$ & $max(Column4)$ \\ \hline
  \end{tabular}
  \\ \textbf{vector_maximum (XMM4)}
\end{center}

\begin{itemize}
\item Se rota el vector de máximos 3 veces y se hacen tres comparaciones, para obtener el máximo general de la matriz:
\end{itemize}

\begin{codesnippet}
  \begin{verbatim}
	146		movdqu rotated_vector_maximum, vector_maximum

	;Rotate vector one píxel to the right and compare.
	;Repeat four times to find the maximum among all píxels in the vector.
	150		pshufb rotated_vector_maximum, mask
	151		pmaxub vector_maximum, rotated_vector_maximum
	152		pshufb rotated_vector_maximum, mask
	153		pmaxub vector_maximum, rotated_vector_maximum
	154		pshufb rotated_vector_maximum, mask
	155		pmaxub vector_maximum, rotated_vector_maximum
	156		jmp .retornarDeMaximos
  \end{verbatim}
\end{codesnippet}

\begin{center}
  \begin{tabular}{|c || c || c || c |}
  \hline
   $max(Column2)$ & $max(Column3)$ & $max(Column4)$ & $max(Column1)$\\ \hline
  \end{tabular}
  \\ \textbf{rotated_vector_maximum (XMM6)}
\end{center}
\begin{center}
    \begin{tabular}{|c || c || c || c |}
  \hline
  $...$ & $...$ & $...$ & $max(Column4, Column1)$ \\ \hline
  \end{tabular}
  \\ \textbf{vector_maximum (XMM4)}
\end{center}
\begin{center}
  \begin{tabular}{|c || c || c || c |}
  \hline
    $max(Column3)$ & $max(Column4)$ & $max(Column1)$ & $max(Column2)$\\ \hline
  \end{tabular}
  \\ \textbf{rotated_vector_maximum (XMM6)}

\end{center}
\begin{center}
  \begin{tabular}{|c || c || c || c |}
  \hline
  $...$ & $...$ & $...$ & $max(Column4, Column1, Column2)$ \\ \hline
  \end{tabular}
  \\ \textbf{vector_maximum (XMM4)}
\end{center}
\begin{center}
  \begin{tabular}{|c || c || c || c |}
  \hline
  $max(Column4)$ & $max(Column1)$ & $max(Column2)$ & $max(Column3)$ \\ \hline
  \end{tabular}
  \\ \textbf{rotated_vector_maximum (XMM6)}
\end{center}
\begin{center}
      \begin{tabular}{|c || c || c || c |}
  \hline
  $...$ & $...$ & $...$ & $max$ \\ \hline
  \end{tabular}
  \\ \textbf{vector_maximum (XMM4)}
\end{center}

\begin{itemize}
\item De la línea 117 a la 134 se obtienen los próximos píxeles a procesar y cómo pasar a la siguiente fila cuando es necesario.
\item Para terminar, se desarma el stack frame.
\end{itemize}

\newpage
\section{Experimentación}
\subsection{Mediciones y metodología de experimentación}

Para poder realizar las mediciones correspondientes a los experimentos realizados utilizamos la instrucción de assembler \textit{rdtsc}. Con ella podemos obtener el valor de Time Stamp Counter (TSC) del procesador -un registro de 64 bits presente en todos los procesadores x86 desde los Pentium-, como este se incrementa en uno con cada ciclo del procesador podemos obtener la duración en cantidad de ticks de una llamada a una función calculando la diferencia de este registro al principio y al final de su ejecución. Ya que esta medida no es constante entre llamadas, por cada caso de test (ejecución de un filtro, con una implementación y para una imagen con un tamaño específico) se realizaron 2500 llamadas. Al mismo tiempo, para poder tener las mediciones individuales de cada iteración y poder calcular la varianza y filtrar outliers se modificó el framework de la cátedra para tener disponible esta información.
La modificación se realizó en el ciclo principal de la función \textit{correr_filtro_imagen} del archivo \textit{tp2.c}:

\begin{codesnippet}
  \begin{verbatim}
    MEDIR_TIEMPO_START(start)
    for (int i = 0; i < config->cant_iteraciones; i++) {
        unsigned long long start_iter, end_iter, iteraciones;
        MEDIR_TIEMPO_START(start_iter)
        aplicador(config);
        MEDIR_TIEMPO_STOP(end_iter)
        iteraciones = end_iter - start_iter;
        fprintf(fp, ";%llu", iteraciones);
    }
    MEDIR_TIEMPO_STOP(end)
  \end{verbatim}
\end{codesnippet}

Luego, una vez que se contó con los tiempos de cada iteración se lidió con los outliers calculando una media 25\% podada. La computadora en la cual se corrieron los experimentos tiene las siguientes especificaciones:

\begin{table}[htb]
  \centering
  \caption{Especificaciones cpu}
  \begin{tabular}{|c|c|}
  \hline
    model name & Intel(R) Core(TM) 14-nm "Broadwell ULT" i5-5350U CPU @ 1.80GHz \\ \hline
    cpu MHz & 1799.948 \\ \hline
    l2 cache size & 256 KB \\ \hline
    l3 cache size & 3072 KB \\ \hline
    mem total & 8 GB 1600 MHz DDR3 \\ \hline
  \hline
  \end{tabular}
  \end{table}

A su vez, cuenta con los flags detallados a continuación: 
fpu vme de pse tsc msr pae mce cx8 apic sep mtrr pge mca cmov pat pse36 clflush mmx fxsr sse sse2 ht syscall nx rdtscp lm constant_tsc rep_good nopl xtopology nonstop_tsc eagerfpu pni pclmulqdq monitor ssse3 cx16 pcid sse4_1 sse4_2 movbe popcnt aes xsave avx rdrand lahf_lm abm 3dnowprefetch invpcid_single fsgsbase avx2 invpcid rdseed flush_l1d \\

Se utilizó la siguiente combinación de sistema operativo y compiladores: Ubuntu/Xenial64 16.04.4 LTS, GCC versión 5.4.0 y NASM versión 2.11.08. Para compilar el código C se usaron los flags \textit{-ggdb -Wall -Wno-unused-parameter -Wextra -std=c99 -pedantic -m64 -O0} y para ASM \textit{-felf64 -g -F dwarf} conforme a los archivos proporcionados por la cátedra de la materia.

La imagen utilizada para todos los experimentos fue la correspondiente a \textit{No Country For Old Men}, proporcionada por la cátedra. Otros detalles de las mediciones se dan específicamente por cada experimento producido.
\newpage
\section{Conclusiones y trabajo futuro}

Gracias a los experimentos realizados en el presente trabajo, se pudo llegar a la conclusión de que las ventajas que puede brindar el paradigma \textbf{SIMD} a la hora de implementar programas que realicen operaciones altamente paralelizables, como el procesamiento de imágenes, son verdaderamente significativas. Esto queda reflejado en la gran brecha de rendimiento que se observa entre algunas implementaciones realizadas con dicho paradigma y las que utilizan el lenguaje de programación C.

Esto siempre debe contraponerse a otro hecho que se hizo presente durante el proceso de implementación: realizar un programa en lenguaje ensamblador resulta, por lo general, considerablemente más difícil que hacerlo en un lenguaje de más alto nivel. El código resultante es menos legible, es más sencillo cometer errores y el proceso de \emph{debugging} se vuelve considerablemente más arduo. Por eso es importante analizar de antemano las características del contexto particular de aplicación, para poder decidir si este esfuerzo adicional realmente vale la pena.

Incluso una vez realizada una implementación en ASM, es necesario comparar el rendimiento de ambas versiones con distintos tamaños de imágenes para ver que efectivamente haya una mejora y el grado de la misma -por ejemplo, en nuestras implementaciones Offset para tamaños grandes obtuvo peores resultados que la versión en C con O3-. Otros aspectos a tener en consideración es el tamaño de los binarios involucrados, que puede ser decisivo en determinados escenarios.

También se intentó la realización de una optimización manual dentro del código ensamblador, con distintos resultados. Por ejemplo: expander los ciclos, inlinear llamadas a funciones, analizar instrucciones individuales para encontrar alternativas, tuvieron distintos beneficios en algunos casos y en otros no. Estas optimizaciones son realizadas por el compilador de C al setear alguna opción del flag O, las distintas técnicas de optimización que realizan los mismos pueden servir de inspiración a la hora de intentar mejorar el rendimiento de un programa, pero su justificación siempre tendrá que ser acompañada de métricas que las respalden.

Por último, pudimos hacer una pequeña prueba con la extensión del modelo de SIMD propuesto por la tecnología AVX2 cuyos resultados confirmaron las ventajas de su utilización en la programación vectorial a bajo nivel.

Esperamos que la experiencia nos provea de nuevos recursos a la hora de afrontar problemáticas similares en el futuro.
\end{document}
